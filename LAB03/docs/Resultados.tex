\section{\normalsize RESULTADOS}
	Após todo esse detalhamento, ambas as implementações foram submetidas à uma bateria de testes. O ambiente de testes possuí as seguintes configurações:
	
\begin{lstlisting}[frame=single,style=base]
@OS@
Linux s-pc 4.19.85-1-MANJARO #1 SMP PREEMPT Thu Nov 21 10:38:39 UTC 2019 x86_64 GNU/Linux

@CPU@
Architecture:                    x86_64
CPU op-mode(s):                  32-bit, 64-bit
Byte Order:                      Little Endian
Address sizes:                   39 bits physical, 48 bits virtual
CPU(s):                          8
On-line CPU(s) list:             0-7
Thread(s) per core:              1
Core(s) per socket:              8
Socket(s):                       1
NUMA node(s):                    1
Vendor ID:                       GenuineIntel
CPU family:                      6
Model:                           158
Model name:                      Intel(R) Core(TM) i7-9700K CPU !!~@~!! 3.60GHz
Stepping:                        12
CPU MHz:                         800.041
CPU max MHz:                     4900,0000
CPU min MHz:                     800,0000
BogoMIPS:                        7202.00
Virtualization:                  VT-x
L1d cache:                       256 KiB
L1i cache:                       256 KiB
L2 cache:                        2 MiB
L3 cache:                        12 MiB
NUMA node0 CPU(s):               0-7

@RAM@
			total			used		free		shared	buff/cache	available
Mem:	32880560	4269840	14884996	220864	13725724	27972388
Swap:	17408220	0				17408220
\end{lstlisting}

	Além disso, o seguinte \textit{script} foi utilizado para automatizar os testes:

\begin{lstlisting}[language=bash]
#!/bin/bash

if [ ! -f "memory" ]; then
	mpicc -o memory mpi_memory_priority.c
fi

if [ ! -d "process" ]; then
	mpicc -o process mpi_process_priority.c
fi

for h in {1..50}; do
	for i in "memory" "process"; do
		for j in {1..16}; do
			for k in 10 100 1000 10000 100000 1000000 10000000 100000000 1000000000 2147483647; do
				mpirun --use-hwthread-cpus -np "$j" "$i" "$k" &>> "$i"_results.txt 
			done
		done
	done
done
\end{lstlisting}

	É importante ressaltar que esse \textit{script}, além de testar a execução das aplicações com diferentes quantidades de processos e tamanhos de vetor, também executa a mesma ação várias vezes, de forma a aumentar a precisão dos resultados. Outro detalhe importante é que o valor 2147483647, passado como último parâmetro na lista de tamanhos de vetores, é exatamente o maior valor que o tipo \textit{int} em C consegue armazenar.
	
	Assim, após várias execuções e uma posterior análise dos resultados, as médias encontradas foram as seguintes:

\begin{sidewaystable}
\begin{tabular}{|p{3cm}|p{1cm}|p{1cm}|p{1cm}|p{1cm}|p{1cm}|p{1cm}|p{1cm}|p{1cm}|p{1cm}|p{1cm}|p{1cm}|p{1cm}|p{1cm}|p{1cm}|p{1cm}|p{1cm}|p{1cm}|p{1cm}|}
\hline
& \multicolumn{16}{|c|}{Número de Processos}\\\hline
Vetor & 1 & 2 & 3 & 4 & 5 & 6 & 7 & 8 & 9 & 10 & 11 & 12 & 13 & 14 & 15 & 16\\\hline
10 & 1 & 2 & 3 & 4 & 5 & 6 & 7 & 8 & 9 & 10 & 11 & 12 & 13 & 14 & 15 & 16\\\hline
100 & 1 & 2 & 3 & 4 & 5 & 6 & 7 & 8 & 9 & 10 & 11 & 12 & 13 & 14 & 15 & 16\\\hline
1000 & 1 & 2 & 3 & 4 & 5 & 6 & 7 & 8 & 9 & 10 & 11 & 12 & 13 & 14 & 15 & 16\\\hline
10000 & 1 & 2 & 3 & 4 & 5 & 6 & 7 & 8 & 9 & 10 & 11 & 12 & 13 & 14 & 15 & 16\\\hline
100000 & 1 & 2 & 3 & 4 & 5 & 6 & 7 & 8 & 9 & 10 & 11 & 12 & 13 & 14 & 15 & 16\\\hline
1000000 & 1 & 2 & 3 & 4 & 5 & 6 & 7 & 8 & 9 & 10 & 11 & 12 & 13 & 14 & 15 & 16\\\hline
10000000 & 1 & 2 & 3 & 4 & 5 & 6 & 7 & 8 & 9 & 10 & 11 & 12 & 13 & 14 & 15 & 16\\\hline
100000000 & 1 & 2 & 3 & 4 & 5 & 6 & 7 & 8 & 9 & 10 & 11 & 12 & 13 & 14 & 15 & 16\\\hline
1000000000 & 1 & 2 & 3 & 4 & 5 & 6 & 7 & 8 & 9 & 10 & 11 & 12 & 13 & 14 & 15 & 16\\\hline
2147483647 & 1 & 2 & 3 & 4 & 5 & 6 & 7 & 8 & 9 & 10 & 11 & 12 & 13 & 14 & 15 & 16\\\hline
\end{tabular}
\end{sidewaystable}
\begin{comment}
{\small
\begin{tabular}{ |p{3cm}|p{1cm}|p{1cm}|p{1cm}|p{1cm}|p{1cm}|p{1cm}|p{1cm}|p{1cm}|p{1cm}|}
\hline
\multicolumn{10}{|c|}{Tamanho do Vetor}\\\hline
10 & 100 & 1000 & 10000 & 100000 & 1000000 & 10000000 & 100000000 & 1000000000 & 2147483647\\\hline
10 & 100 & 1000 & 10000 & 100000 & 1000000 & 10000000 & 100000000 & 1000000000 & 2147483647\\\hline
\end{tabular}
}
\end{comment}