\section{\normalsize MPI (\textit{MESSAGE PASSING INTERFACE})}
	O  MPI é um padrão de interface para a troca de mensagens em máquinas paralelas com memória distribuída. Apesar de alguns pensarem dessa forma, o MPI não é um compilador ou um produto específico.
	
	No padrão MPI, uma aplicação é constituída por um ou mais processos que se comunicam, acionando-se funções para o envio e recebimento de mensagens entre os processos. Inicialmente, na maioria das implementações, um conjunto fixo de processos é criado. Porém, esses processos podem executar diferentes programas. Por isso, o padrão MPI é algumas vezes referido como MPMD (\textit{Multiple Program Multiple Data}).

Elementos importantes em implementações paralelas são a comunicação de dados entre processos paralelos e o balanceamento da carga. É importante frisar que o número de processos no MPI normalmente é fixo. Dito isso, tais processos podem usar mecanismos de comunicação ponto a ponto (operações para enviar mensagens de um determinado processo a outro), ou coletivas, na qual um grupo de processos pode invocar operações coletivas de comunicação para executar operações globais. 

Sobre o MPI, o mesmo é capaz de suportar comunicação assíncrona e programação modular, através de mecanismos de comunicadores que permitem ao usuário MPI definir módulos que encapsulem estruturas de comunicação interna.

Finalmente, os algoritmos que criam um processo para cada processador podem ser implementados, diretamente, utilizando-se comunicação ponto a ponto ou coletivas, sendo que os algoritmos que implementam a criação de tarefas dinâmicas ou que garantem a execução concorrente de muitas tarefas, num único processador, precisam de um refinamento nas implementações com o MPI.