\section{\normalsize SOBRE A IMPLEMENTAÇÃO}
	
	\subsection{\normalsize SOBRE A SOLUÇÃO GENÉRICA DO PROBLEMA}
	Essa sessão será dedicada à explicação da implementação do algoritmo de ordenação paralelo.
	
	Primeiramente, através do \textit{input} do usuário\footnote{passado como parâmetro na execução da aplicação}, é definido o tamanho do vetor. Em seguida, após alocar memória para esse vetor, o mesmo é preenchido com números aleatórios. 
	
	Aqui é importante frisar que tanto a versão paralela quanto a sequencial possuem os mesmos dados alocados nas mesmas posições de seus respectivos vetores. Isso foi feito para garantir que os resultados finais entre a versão paralela e sequencial não sejam afetados pelos elementos nos vetores de cada versão.
	
	\begin{lstlisting}[style=C]
srand(time(NULL));
for (i = 0; i < size; i++) { //size é o tamanho do vetor
	int random = rand() / 1000;
	parallel_array[i] = random;
	sequential_array[i] = random;
}
\end{lstlisting}
	
	Como dito na sessão \ref{definitions}, o algoritmo de ordenação utilizado foi o \textit{Bucket Sort}\footnote{usado para classificação} junto com a função \textit{qsort()}\footnote{usada para ordenação}. Para garantir que as versões paralela e sequencial criassem e ordenassem o mesmo número de \textit{buckets} usando o mesmo método de inserção, foi definido que o número de \textit{buckets} é equivalente ao número de processos disponíveis na versão paralela (incluso o mestre).
	
	Isso foi feito para garantir que a versão sequencial não obtivesse um desempenho inferior por ficar, recursivamente, dividindo o vetor original em \textit{buckets}, o que implica em alocar memória e inserir, através do cálculo apresentado na sessão \ref{bucket}, os elementos em cada novo \textit{bucket} criado.
		
	Além disso, para garantir a precisão dos resultados, tanto a versão paralela quanto a sequência tem, em suas implementações, a tarefa de encontrar o maior e o menor elemento em seus respectivos vetores, mesmo que esses elementos sejam, no final das contas, os mesmos\footnote{dado o fato que ambos os vetores são iguais}.
	
	De resto, tanto a versão paralela quanto a sequencial tratam de dividir o vetor original em \textit{buckets}, ordenar tais \textit{buckets} e depois reagrupá-los no vetor original, sendo que a diferença fundamental entre elas é que, enquanto na versão paralela cada \textit{bucket} é ordenado por um processo, na versão sequencial os \textit{buckets} são ordenados sequencialmente por um único processo. Apesar dessa ser a única diferença fundamental entre ambas as versões, a versão paralela possuí algumas características à mais que valem a pena serem comentadas.
	
		\subsubsection{\normalsize IMPLEMENTAÇÃO PARALELA}
			Primeiramente, como era de se esperar, apenas o nodo mestre possuí o vetor original. Esse nodo também é o responsável, ao final da execução paralela, por executar a versão sequencial do algoritmo. Isso é garantido através da condicional:
			\newpage
			\begin{lstlisting}[style=C] 			
if(rank = = 0) {
	criação e população dos vetores (paralelo e sequencial)
	
	parallel_sort(...); // função que executa a versão paralela do algoritmo
	...
	sequential_sort(...); /* função que executa a versão sequencial do 
	algoritmo. Por estar dentro da condicional, existe a garantia de 
	que apenas um processo irá executar essa função. */
}
\end{lstlisting}
			
			O nodo mestre é o único responsável por criar os \textit{buckets} e inserir os elementos neles. Para isso, o mesmo possuí duas variáveis, sendo uma delas um vetor de inteiros\footnote{a partir desse ponto, referenciado como \textit{bucket\_sizes}} cujo tamanho é equivalente ao número de processos existentes(incluso o nodo mestre), e a outra sendo um vetores com ponteiros para inteiros\footnote{a partir desse ponto referenciado como \textit{*short\_bucket}.\\Perceba que o símbolo * permanece, isso foi feito para ser um constante lembrete para o leitor de que tal variável se trata de um vetor de ponteiros}, novamente sendo esse vetor com tamanho equivalente ao número de processos.
			
			
			O \textit{bucket\_sizes} serve para armazer o tamanho do \textit{bucket} que cada processo irá receber. Em outras palavras, cada posição desse vetor(0 até \textit{n}) é corresponde à um processo. Já o \textit{*short\_bucket} é quem irá armazenar os \textit{buckets} e seus elementos, sendo que a mesma lógica do \textit{bucket\_sizes} se aplica à ele.
			
			Dessa forma, tem-se que o processo 1 irá ordenar o \textit{bucket} armazenado na posição 1 do vetor \textit{*short\_bucket}, sendo que o tamanho desse vetor está armazenado na posição 1 do vetor \textit{bucket\_sizes}.
			
			Dados esse dados, primeiramente o nodo mestre irá enviar para os processos 1 até \textit{n}, o tamanho de vetor que os mesmos irão receber e ordenar. Após o mestre receber a confirmação de que cada nodo escravo\footnote{a partir desse ponto, o termo nodo mestre e/ou processo mestre será substituído por apenas mestre e o(s) termo(s) nodo(s) escravo(s) e/ou processo(s) escravo(s) irá(ão) ser substituído(s) por escravo(s)} recebeu sua mensagem, o mesmo, então, envia para os escravos seus respectivos \textit{buckets}\footnote{\textit{*short\_bucket}[1..\textit{n}]}. Após ter enviado para cada escravo seu respectivo \textit{bucket}, o mestre então trata de ordenar o \textit{bucket} 0. 

			Note que essa estratégia impede que o mestre fique ocioso, esperando o retorno dos escravos para só então prosseguir com a concatenação dos \textit{buckets}.	Ao invés disso, como desde o princípio o mestre foi incluído no cálculo do tamanho e inserção de elementos nos \textit{buckets}, o mesmo, após enviar todas as informações necessárias para os escravos começarem seu processamento, trata de ordenar um dos \textit{buckets}, o que, além de impedir a ociosidade do mestre, acrescenta ao rol de processos mais um para realizar o processamento dos dados, o que possibilita um aumento, pequeno ou grande\footnote{esse aumento depende do tamanho do vetor original}, na velocidade de resposta do algoritmo.
		
			O código abaixo exemplifica esse processo:			
			\newpage
			\begin{lstlisting}[style=C]		
for (i = 1; i < numprocs; i++) { /* numprocs aqui representa o número total de processos existentes. */

	/* Mestre enviando para os escravos 1 à (numprocs - 1) a informação de qual será o tamanho do bucket que o mesmo irá receber. */
	MPI_Send(&bucket_sizes[i], 1, MPI_INT, i, 0, MPI_COMM_WORLD); 

	/* Mestre enviado para os escravos seus respectivos buckets.
	Para maiores informações sobre a função MPI_Send favor rever a seção !\textcolor{mGreen}{\ref{send}}! */
	MPI_Send(short_bucket[i], bucket_sizes[i], MPI_INT, i, 0,MPI_COMM_WORLD);
	
	quick_sort(short_bucket[0], bucket_sizes[0]); // Ordenação do bucket 0
}
\end{lstlisting}
			
			Agora, no lado dos escravos, os mesmos inicialmente permanecem bloqueados, aguardando que o mestre envie o tamanho de seus respectivos \textit{buckets}. Após receberem essa informação, cada escravo aloca memória para um ponteiro de inteiros cujo tamanho será aquele recebido anteriormente. Feito isso, os escravos voltam a esperar que o mestre os envie seus respectivos \textit{buckets}, que, ao serem recebidos, serão armazenas no ponteiro anteriormente alocado.
			
			Nesse ponto, os escravos já possuem todas as informações necessárias para iniciarem seu trabalho, então, cada um deles trata de ordenar seu \textit{bucket}. Assim que um escravo termina de ordenar seu \textit{bucket}, o mesmo trata de enviá-lo para o mestre. Assim que recebe a confirmação de que o mestre recebeu o \textit{bucket}, o escravo é finalizado.
			
			O código abaixo exemplifica esse processo:
			\begin{lstlisting}[style=C]		
if(rank = = 0) {
	// Execução do mestre
	...
} else {
	MPI_Recv(&size, 1, MPI_INT, 0, 0, MPI_COMM_WORLD, &status); 
	/* Escravo recebe do mestre a informação  de qual o tamanho de seu bucket e armazena essa informação na variável size.
	
	Para maiores informações sobre a função MPI_Recv favor rever a seção !\textcolor{mGreen}{\ref{recv}}! */
	
	int *bucket = (int *)malloc(sizeof(int) * size); 
	/* Escravo aloca uma área de memória para armazenar seu bucket quando o receber do mestre */ 

	MPI_Recv(bucket, size, MPI_INT, 0, 0, MPI_COMM_WORLD, &status); 
	/* Escravo recebe do mestre seu bucket e o copia para a área de memória previamente criada.*/

    quick_sort(bucket, size); /* Escravo ordena seu bucket*/

    MPI_Send(bucket, size, MPI_INT, 0, 0, MPI_COMM_WORLD); /* E o envia, já ordenado, de volta para o mestre */
}
\end{lstlisting}
			
			Voltando para o mestre, após terminar de ordenar seu \textit{bucket}, o mesmo irá aguardar receber dos escravos os \textit{buckets} para eles enviados. É importante ressaltar que o mestre recebe os \textit{buckets} na mesma ordem em que ele os enviou, ou seja, primeiramente o mestre recebe o \textit{bucket} enviado ao processo 1, por mais que outro processo qualquer tenha terminado sua tarefa mais cedo e esteja a mais tempo tentando enviar seu \textit{bucket} para o mestre. Além disso, o mestre não apenas recebe os \textit{buckets} na ordem na qual os envio, como também armazena cada vetor recebido na exata posição do vetor \textit{short\_bucket} no qual o mesmo estava originalmente. Isso é possível graças ao fato de que cada processo recebe o \textit{bucket} equivalente à seu rank e, portanto, ao enviar o \textit{bucket} de volta ao mestre, o mesmo só precisa usar o rank do emissor da mensagem para determinar a posição do \textit{bucket} recebido no vetor de \textit{buckets} original.
			
			Finalmente, após receber todos os \textit{buckets}, o mestre concatena os \textit{buckets} em um único vetor, que, graças ao método de inserção do \textit{Bucket Sort}, melhor explicado na seção \ref{classification}, já estará ordenado. Feito tudo isso, o mestre calcula quanto tempo durou todo esse processamento e, então, passa a executar, sozinho, a versão sequencial do programa.
			
			O código abaixo exemplifica esse processo:
			\begin{lstlisting}[style=C]
for (i = 1; i < numprocs; i++) {
	MPI_Recv(short_bucket[i], bucket_sizes[i], MPI_INT, i, 0, MPI_COMM_WORLD, &status);
	/* Recebe o bucket de cada escravo e o armazena em sua posição 
	correspondente do vetor de buckets original.
	*/
}

/* Percorre cada elemento de cada bucket, em ordem, e o adiciona em sua posição final no vetor original */
int index = 0;
for (i = 0; i < numprocs; i++) {
	for (j = 0; j < bucket_sizes[i]; j++) {
    		parallel_array[index++] = short_bucket[i][j];
	}
}
\end{lstlisting}

	\subsection{\normalsize SOBRE AS SOLUÇÕES PERSONALIZADAS DO PROBLEMA}\label{personal}
		Conforme mencionado na seção \ref{definitions}, foram realizadas duas implementações da aplicação. Em ambas foram implementadas a versão sequencial e paralela. Além disso, o funcionamento de ambas as implementações, em síntese, é o mesmo apresentado na seção anterior a essa. A diferença entre essas implementações consiste no modo como os \textit{buckets} são alocados.
		
		A primeira implementação\footnote{contida no arquivo \textbf{mpi\_process\_priority.c}}, aloca, para cada \textit{bucket}, o tamanho equivalente ao vetor original. Ao fazer isso, adquire-se um pequeno desempenho, uma vez que, mesmo não sabendo qual o tamanho de cada \textit{bucket}, se sabe que o mesmo não pode ser maior que o vetor original. Assim, é garantido que o cálculo para inserir os elementos nos \textit{buckets}, que, conforme explicado na seção \ref{complex}, possuí complexidade  $\theta(n)$, será executado apenas uma única vez. Essa solução, apesar de aumentar o desempenho da aplicação, se torna inviável em casos no qual o vetor é extremamente grande e existem dois ou mais processos, uma vez que a memória necessária para alocar os \textit{buckets} pode ser superior à memória real da máquina.
		
		A segunda implementação\footnote{contida no arquivo \textbf{mpi\_memory\_priority.c}}, aloca a memória dos \textit{buckets} apenas após saber o tamanho exato de cada \textit{bucket}. Para isso, o cálculo de inserção é executado duas vezes, a primeira para determinar o tamanho dos \textit{buckets}, e a segunda para alocar os elementos em seus respectivos \textit{buckets}. Ao tratar a alocação de \textit{buckets} dessa forma, observa-se uma notável perda de desempenho da aplicação, mas ao mesmo tempo, se torna possível realizar a ordenação de vetores maiores do que o suportado pela primeira implementação. Essa solução é simplesmente essencial em sistemas que operam com pouca memória, mas é completamente inútil em situações nas quais o sistema possuí uma quantidade absurda de memória e/ou o tamanho do vetor não é extremamente grande.

		O código abaixo exemplifica isso:
		\begin{lstlisting}[style=C]
/* Implementação que aloca cada bucket com o tamanho do vetor original */
for (i = 0; i < numprocs; i++) { // numprocs é o número total de buckets
	short_bucket[i] = (int *)malloc(sizeof(int) * size); // size é o tamanho total do vetor original.
}

/* Implementação que aloca cada bucket com seu tamanho exato 
	Envolve calcular primeiro a inserção dos elementos 
	para determinar o tamanho dos buckets para,
	só depois, alocar a memória e recalcular a inserção
	de elementos	
*/

unsigned int max = get_max(array, size);
unsigned int min = get_min(array, size);

for (i = 0; i < numprocs; i++) {
	bucket_sizes[i] = 0;


double pivot = (((double)max - min + 1) / (numprocs));
for(i = 0; i < size; i++) {
    	int idx = (array[i] - min) / pivot;
    	bucket_sizes[idx]++;
}

for (i = 0; i < numprocs; i++) {
	short_bucket[i] = (int *)malloc(sizeof(int) * bucket_sizes[i]);
}
\end{lstlisting}
		
	\subsection{\normalsize DEMAIS CONSIDERAÇÕES}
		Algumas observações que são importantes frisar:
		
		\begin{itemize}
			\item O código fonte contém alguns comentários que explicam, de forma breve, as partes principais da aplicação. Tais comentário, para facilitar a localização, estão no estilo de múltiplas linhas(\textit{/*comentário*/}).
			
			\item Com o intuito de facilitar a leitura do código fonte, todas as partes do código que foram possíveis de serem convertidas em funções, o foram.
			
			\item O código abaixo explica como foi implementado o cálculo de inserção dos elementos nos \textit{buckets}:\\
			\begin{lstlisting}[style=C] 
unsigned int max = get_max(array, size); // Encontra o maior elemento do vetor.
unsigned int min = get_min(array, size); // E o menor

// Calcula qual é o melhor pivo
double pivot = (((double)max - min + 1) / (numprocs)); // numprocs é a quantidade de buckets
for (i = 0; i < size; i++) {
	int index = (array[i] - min) / pivot; // Encontra o índice do bucket no qual esse elemento será inserido
	short_bucket[index][bucket_sizes[index]] = array[i]; // Insere o elemento no bucket
	bucket_sizes[index]++;
}
\end{lstlisting}
		\end{itemize}