\section{\normalsize SOBRE A IMPLEMENTAÇÃO}
	Essa sessão será dedicada à explicação da implementação do algoritmo de ordenação paralelo.
	
	Primeiramente, através do \textit{input} do usuário\footnote{passado como parâmetro na execução da aplicação}, é definido o tamanho do vetor. Em seguida, após alocar memória para esse vetor, o mesmo é preenchido com números aleatórios. 
	
	Aqui é importante frisar que tanto a versão paralela quanto a sequencial possuem os mesmos dados alocados nas mesmas posições de seus respectivos vetores. Isso foi feito para garantir que os resultados finais entre a versão paralela e sequencial não sejam afetados pelos elementos nos vetores de cada versão.
	
	\begin{lstlisting}[style=C]
srand(time(NULL));
for (i = 0; i < size; i++) { //size é o tamanho do vetor
	int random = rand() / 1000;
	parallel_array[i] = random;
	sequential_array[i] = random;
}
	\end{lstlisting}
	
	 
	Como dito na sessão \ref{definitions}, o algoritmo de ordenação utilizado foi o \textit{Bucket Sort}. Para implementá-lo 