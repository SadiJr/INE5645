\section{\normalsize DEFINIÇÃO DO ESTUDO}
	O presente relatório buscará analisar o desempenho de uma aplicação responsável por ordenar um vetor de inteiros, cujo tamanho é definido pelo usuário e  cujos valores são gerados de forma aleatória, em sua implementação paralela em comparação com sua implementação sequencial.
	
	Para isso, os seguintes critérios foram seguidos:
	\begin{itemize}
		\item A aplicação foi inteiramente escrita na linguagem C.
		\item Foi utilizada a biblioteca OpenMPI para realizar a comunicação entre os diferentes processos.
		\item O algoritmo de ordenação usado foi o \textit{bucket sort}, para a divisão dos dados, e o \textit{quick sort} para a real ordenação dos dados.
		\item Tanto a versão paralela quanto a versão sequencial ordenam o mesmo conjunto de dados (copiados para ambas as versões) e utilizam o mesmo número de \textit{buckets}, com o objetivo de os resultados não serem prejudicados por diferenças nas implementações\footnote{afora o paralelismo, é claro.}. 
		\item Foram escritos duas aplicação para realizar essa análise, uma delas dando prioridade para o processamento dos dados e outra priorizando a otimização de memória.
	\end{itemize}
	
	Tais critérios serão melhor explicados ao longo do relatório.