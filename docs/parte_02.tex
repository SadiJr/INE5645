\section{\normalsize ALGORITMOS}
	Os algoritmos de ordenação usados neste trabalho foram:
	
	\subsection{\normalsize BUBBLE SORT}
		O Bubble Sort é um algoritmo simples que é usado para ordenar um dado conjunto de \textit{n} elementos fornecidos na forma de um \textit{array} com \textit{n} número de elementos. O Bubble Sort compara todos os elementos, um por um, e classifica-os com base em seus valores.

		Se a matriz determinada tiver que ser classificada em ordem crescente, então a ordenação em bolha começará comparando o primeiro elemento da matriz com o segundo, se o primeiro elemento for maior que o segundo, ele trocará os elementos e, em seguida, siga em frente para comparar o segundo e o terceiro elemento, e assim por diante.

		Se tivermos \textit{n} elementos totais, precisamos repetir esse processo por \textit{n-1} vezes.

		É conhecido como Bubble sort, porque a cada iteração completa é o maior elemento na matriz dada, borbulha em direção ao último lugar ou ao índice mais alto, assim como uma bolha d'água sobe até a superfície da água.

		A classificação ocorre percorrendo todos os elementos um por um e comparando-o com o elemento adjacente e trocando-os, se necessário.

		No melhor caso, o algoritmo executa \textit{n} operações relevantes, onde \textit{n} representa o número de elementos do vetor. No pior caso, são feitas \textit{n}$^{2}$ operações. A complexidade desse algoritmo é de ordem quadrática. Por isso, ele não é recomendado para programas que precisem de velocidade e operem com quantidade elevada de dados.
	
	\subsection{\normalsize MERGE SORT}
		A ideia básica do Merge Sort é criar uma sequência ordenada a partir de duas outras também ordenadas. Para isso, o algoritmo Merge Sort divide a sequência original em pares de dados, agrupa estes pares na ordem desejada; depois as agrupa as sequências de pares já ordenados, formando uma nova sequência ordenada de quatro elementos, e assim por diante, até ter toda a sequência ordenada.\\
		O algoritmo segue três passos comuns aos algoritmos dividir-para-conquistar:
		
		\begin{enumerate}
			\item Dividir:\\
				Dividir os dados em subsequências pequenas. Este passo é realizado recursivamente, iniciando com a divisão do vetor de \textit{n} elementos em duas metades, cada uma das metades é novamente dividida em duas novas metades e assim por diante, até que não seja mais possível a divisão (ou seja, sobrem \textit{n} vetores com um elemento cada).
				
			\item Conquistar:\\
				Classificar as duas metades recursivamente aplicando o algoritmo do Merge Sort;
			
			\item Combinar:\\
				Juntar as duas metades em um único conjunto já classificado. Para completar a ordenação do vetor original de \textit{n} elementos, faz-se o Merge ou a fusão dos sub-vetores já ordenados.
		\end{enumerate}
		
		A desvantagem do Merge Sort é que requer o dobro de memória, ou seja, precisa de um vetor com as mesmas dimensões do vetor que está sendo classificado.
		
		A vantagem (ou desvantagem, dependendo do observador), é que a complexidade do Merge Sort (O(n log n)) é a mesma para o melhor, médio e pior caso, uma vez que, independente da situação dos dados no vetor, o algoritmo irá sempre dividir e intercalar os dados.