\section{\normalsize CONCLUSÕES}
	
	\subsection{\normalsize CONCLUSÕES A RESPEITO DA LINGUAGEM \textit{C}}
		Com relação à \textit{threads}, o limite máximo de \textit{threads} possíveis de serem executadas sem erro, na máquina anteriormente detalhada, é de 30.000 \textit{threads}. É possível executar com até 33.010 \textit{threads}, porém, se esse valor for utilizado, apesar de nenhum erro ocorrer, a aplicação não consegue finalizar, ficando ``congelada''.
		
		Agora com relação aos processos, o limite máximo de processos possíveis de serem executados sem erro é de 10.330 processos.
		
		Finalmente, de forma geral e esperada, os algoritmos de ordenação executados em C apresentaram uma velocidade maior na ordenação das matrizes em comparação com os algoritmos executados em Java, tendo em mente que as matrizes usadas por ambas as linguagens possuíam o mesmo tamanho, embora não necessariamente possuíssem os mesmos dados.
		 
	\subsection{\normalsize CONCLUSÕES A RESPEITO DA LINGUAGEM \textit{Java}}
		Em Java, o limite máximo de \textit{threads} se mostrou superior em comparação à linguagem C, onde foi possível, embora com enorme perda de desempenho, executar 1.000.000 \textit{threads}, tendo em mente que também se mostrou possível executar um número maior de \textit{threads}, o que me leva a concluir que não existem limites para o número máximo de \textit{threads} em Java, ou que, pelo menos, esse limite é extremamente alto.
	
	\subsection{\normalsize CONCLUSÕES A RESPEITO DO ALGORITMO \textit{BUBBLE SORT}}
		O algoritmo apresenta um desempenho inferior ao \textit{Merge Sort} e, conforme o tamanho da matriz aumenta, o desempenho do algoritmo diminuí (considerando o mesmo número de \textit{threads}). 
		
		Agora, com relação a linguagem, ao manter o mesmo tamanho de matriz e o mesmo número de \textit{threads} em ambas as aplicações, notou-se um claro desempenho superior da linguagem C em relação à linguagem Java. Como, por exemplo, em uma matriz 100x100 e com 10 \textit{threads} executando, em C a ordenação da matriz levou 0.018756 segundos, enquanto em Java a mesma ordenação levou 0.047326123 segundos. 
		
		Para maiores detalhes e mais casos de exemplo, favor ver os \textit{logs} de cada aplicação.
		
	\subsection{\normalsize CONCLUSÕES A RESPEITO DO ALGORITMO \textit{MERGE SORT}}
		O algoritmo apresenta um desempenho muito bom, mesmo com matrizes enormes e poucas \textit{threads}. Novamente a superioridade de desempenho da linguagem C sobre a linguagem Java se faz presente. Tomando como exemplo o caso relatado acima, enquanto a ordenação em C levou 0.003700 segundos, a ordenação em Java levou 0.010905595 segundos, o que, apesar de parecer ser pouco, é uma diferença colossal de desempenho.
		
	\subsection{\normalsize DEMAIS CONSIDERAÇÕES}
		Como era de se esperar, quanto maior o tamanho da matriz, mais tempo se leva para ordena-lá. Além disso, percebeu-se que não vale a pena criar mais \textit{threads} do que o limite físico de núcleos da máquina, uma vez que o ganho de desempenho ao fazer isso, quando existe, não é muito significante, além, é claro, de poder ter o efeito oposto e acabar diminuindo o desempenho da aplicação.
		
		Finalmente, pode-se perceber que o desempenho da aplicação aumenta se a matriz tiver poucos linhas, independente do tamanho das colunas, em detrimento a uma matriz com muitas linhas mas colunas pequenas.
		
		Por fim, mas não menos importante, vale ressaltar que, caso reste alguma dúvida com relação ao desempenho das aplicações, basta ler os \textit{logs} anexados e, caso isso não seja suficiente, basta executar as aplicações anexadas você mesmo e fazer seus próprios testes e tirar suas próprias conclusões.