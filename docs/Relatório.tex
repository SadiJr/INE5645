\documentclass[12pt,a4paper,brazil,abntex2]{article}
\usepackage[utf8]{inputenc}
\usepackage[portuguese]{babel}
\usepackage[T1]{fontenc}
\usepackage{amsmath}
\usepackage{amsfonts}
\usepackage{amssymb}
\usepackage{makeidx}
\usepackage{graphicx}
%\usepackage{lmodern}			% Usa a fonte Latin Modern			
\usepackage{indentfirst}		% Indenta o primeiro parágrafo de cada seção.
\usepackage{microtype} 			% para melhorias de justificação
\usepackage[left=3cm,right=2cm,top=3cm,bottom=2cm]{geometry}
\usepackage{url}
\usepackage[hidelinks, breaklinks=true]{hyperref}
\usepackage{setspace}
\usepackage{cite}
\usepackage{float}
\usepackage{breakurl}
\usepackage{verbatim}
\usepackage{listings}
\usepackage{xcolor}

\begin{document}

\singlespacing
\begin{titlepage}
\begin{center}
\begin{figure}[!htb]
\center

\includegraphics[scale=0.25]{/home/sadi/Downloads/Curso/Brasao/Sigla.pdf} 

\end{figure}
{\bf  UNIVERSIDADE FEDERAL DE SANTA CATARINA}\\[0.2cm]
{\bf CENTRO TECNOLÓGICO}\\[0.2cm]
{\bf  DEPARTAMENTO DE INFORMÁTICA E ESTATÍSTICA}\\[5.5cm]
{\bf \large RELATÓRIO DE ANÁLISE DE DESEMPENHO USANDO PARALELISMO}\\[3.6 cm]
{Sadi Júnior Domingos Jacinto}\\[1cm]
{Professor orientador: Odorico Machado Mendizabal}\\[4.1 cm]
{Florianópolis}\\[0.2cm]
{2019}
\newpage
\thispagestyle{empty}
{Sadi Júnior Domingos Jacinto}\\[9cm]
{\bf \large RELATÓRIO DE ANÁLISE DE DESEMPENHO USANDO PARALELISMO}\\[0.5cm]
    \begin{flushright}
    \begin{list}{}{
      \setlength{\leftmargin}{7.2cm}
      \setlength{\rightmargin}{0cm}
      \setlength{\labelwidth}{0pt}
      \setlength{\labelsep}{\leftmargin}}
      \item Análise de desempenho de algoritmos de ordenação usando paralelismo requerido pelo professor da disciplina Programação Paralela e Distribuída, Odorico Machado Mendizabal, necessário para obtenção de nota.\\[0.2 cm] 
      \setlength{\labelsep}{\leftmargin}
      \item Professor orientador: Odorico Machado Mendizabal\
      \\[8.2cm]
     \end{list}
	 \end{flushright}
{Florianópolis}\\[0.2cm]
{2019}
\end{center}
\end{titlepage} %término da capa

\newpage
\thispagestyle{empty}
\begin{center}
\tableofcontents
\end{center}

%\definecolor{blue(pigment)}{RGB}{85, 85, 255}

\lstdefinestyle{base}{
  language=bash,
  emptylines=1,
  breaklines=true,
  basicstyle=\ttfamily\color{black},
  moredelim=**[is][\color{red}]{@}{@},
  moredelim=**[is][\color{black}]{!!}{!!}
}
	
\section{\normalsize PROBLEMÁTICA}
	O presente trabalho refere-se à análise
\section{\normalsize ALGORITMOS}
\section{\normalsize AMBIENTE DE TESTE}
\section{\normalsize CONCLUSÕES}
\section{\normalsize CONCLUSÕES}
\end{document}
