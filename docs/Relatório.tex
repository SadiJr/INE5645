\documentclass[12pt,a4paper,brazil,abntex2]{article}
\usepackage[utf8]{inputenc}
\usepackage[portuguese]{babel}
\usepackage[T1]{fontenc}
\usepackage{amsmath}
\usepackage{amsfonts}
\usepackage{amssymb}
\usepackage{makeidx}
\usepackage{graphicx}
%\usepackage{lmodern}			% Usa a fonte Latin Modern			
\usepackage{indentfirst}		% Indenta o primeiro parágrafo de cada seção.
\usepackage{microtype} 			% para melhorias de justificação
\usepackage[left=3cm,right=2cm,top=3cm,bottom=2cm]{geometry}
\usepackage{url}
\usepackage[hidelinks, breaklinks=true]{hyperref}
\usepackage{setspace}
\usepackage{cite}
\usepackage{float}
\usepackage{breakurl}
\usepackage{verbatim}
\usepackage{listings}
\usepackage{xcolor}

\begin{document}

\singlespacing
\begin{titlepage}
\begin{center}
\begin{figure}[!htb]
\center

\includegraphics[scale=0.25]{/home/sadi/Downloads/Curso/Brasao/Sigla.pdf} 

\end{figure}
{\bf  UNIVERSIDADE FEDERAL DE SANTA CATARINA}\\[0.2cm]
{\bf CENTRO TECNOLÓGICO}\\[0.2cm]
{\bf  DEPARTAMENTO DE INFORMÁTICA E ESTATÍSTICA}\\[5.5cm]
{\bf \large RELATÓRIO DE ANÁLISE DE DESEMPENHO USANDO PARALELISMO}\\[3.6 cm]
{Sadi Júnior Domingos Jacinto}\\[1cm]
{Professor orientador: Odorico Machado Mendizabal}\\[4.1 cm]
{Florianópolis}\\[0.2cm]
{2019}
\newpage
\thispagestyle{empty}
{Sadi Júnior Domingos Jacinto}\\[9cm]
{\bf \large RELATÓRIO DE ANÁLISE DE DESEMPENHO USANDO PARALELISMO}\\[0.5cm]
    \begin{flushright}
    \begin{list}{}{
      \setlength{\leftmargin}{7.2cm}
      \setlength{\rightmargin}{0cm}
      \setlength{\labelwidth}{0pt}
      \setlength{\labelsep}{\leftmargin}}
      \item Análise de desempenho de algoritmos de ordenação usando paralelismo requerido pelo professor da disciplina Programação Paralela e Distribuída, Odorico Machado Mendizabal, necessário para obtenção de nota.\\[0.2 cm] 
      \setlength{\labelsep}{\leftmargin}
      \item Professor orientador: Odorico Machado Mendizabal\
      \\[8.2cm]
     \end{list}
	 \end{flushright}
{Florianópolis}\\[0.2cm]
{2019}
\end{center}
\end{titlepage} %término da capa

\newpage
\thispagestyle{empty}
\begin{center}
\tableofcontents
\end{center}

%\definecolor{blue(pigment)}{RGB}{85, 85, 255}

\lstdefinestyle{base}{
  language=bash,
  emptylines=1,
  breaklines=true,
  basicstyle=\ttfamily\color{black},
  moredelim=**[is][\color{red}]{@}{@},
  moredelim=**[is][\color{black}]{!!}{!!}
}
	
\section{\normalsize PROBLEMÁTICA}
	O presente trabalho refere-se à análise
\section{\normalsize ALGORITMOS}
	Os algoritmos de ordenação usados neste trabalho foram:
	
	\subsection{\normalsize \textit{BUBBLE SORT}}
		O Bubble Sort é um algoritmo simples que é usado para ordenar um dado conjunto de \textit{n} elementos fornecidos na forma de um \textit{array} com \textit{n} número de elementos. O Bubble Sort compara todos os elementos, um por um, e classifica-os com base em seus valores.

		Se a matriz determinada tiver que ser classificada em ordem crescente, então a ordenação em bolha começará comparando o primeiro elemento da matriz com o segundo, se o primeiro elemento for maior que o segundo, ele trocará os elementos e, em seguida, siga em frente para comparar o segundo e o terceiro elemento, e assim por diante.

		Se tivermos \textit{n} elementos totais, precisamos repetir esse processo por \textit{n-1} vezes.

		É conhecido como Bubble sort, porque a cada iteração completa o maior elemento na matriz dada borbulha em direção ao último lugar ou ao índice mais alto, assim como uma bolha d'água sobe até a superfície da água.

		A classificação ocorre percorrendo todos os elementos um por um e comparando-o com o elemento adjacente e trocando-os, se necessário.

		No melhor caso, o algoritmo executa \textit{n} operações relevantes, onde \textit{n} representa o número de elementos do vetor. No pior caso, são feitas \textit{n}$^{2}$ operações. A complexidade desse algoritmo é de ordem quadrática. Por isso, ele não é recomendado para programas que precisem de velocidade e operem com quantidade elevada de dados.
	
	\subsection{\normalsize \textit{MERGE SORT}}
		A ideia básica do Merge Sort é criar uma sequência ordenada a partir de duas outras também ordenadas. Para isso, o algoritmo Merge Sort divide a sequência original em pares de dados, ordena estes pares na ordem desejada, depois reagrupa as sequências de pares já ordenados, formando uma nova sequência ordenada dos elementos, repetindo esse processa até ter toda a sequência ordenada.\\
		O algoritmo segue três passos comuns aos algoritmos dividir-para-conquistar:
		
		\begin{enumerate}
			\item Dividir:\\
				Dividir os dados em subsequências pequenas. Este passo é realizado recursivamente, iniciando com a divisão do vetor de \textit{n} elementos em duas metades, cada uma das metades é novamente dividida em duas novas metades e assim por diante, até que não seja mais possível a divisão (ou seja, sobrem \textit{n} vetores com um elemento cada).
				
			\item Conquistar:\\
				Classificar as duas metades recursivamente aplicando o algoritmo do \textit{Merge Sort}.
			
			\item Combinar:\\
				Juntar as duas metades em um único conjunto já classificado. Para completar a ordenação do vetor original de \textit{n} elementos, faz-se o \textit{merge} ou a fusão dos sub-vetores já ordenados.
		\end{enumerate}
		
		A desvantagem do \textit{Merge Sort} é que requer o dobro de memória, ou seja, precisa de um vetor com as mesmas dimensões do vetor que está sendo classificado.
		
		A vantagem (ou desvantagem, dependendo do observador), é que a complexidade do \textit{Merge Sort} (\textit{O}(\textit{n log n})) é a mesma para o melhor, médio e pior caso, uma vez que, independente da situação dos dados no vetor, o algoritmo irá sempre dividir e intercalar os dados.
\section{\normalsize AMBIENTE DE TESTE}
\section{\normalsize METODOLOGIA}
	Para tornar o processo paralelo possível, ambas as aplicações (em C e em Java), utilizam o \textit{input} do usuário para definir o número de linhas e colunas 
\section{\normalsize CONCLUSÕES}
	
	\subsection{\normalsize CONCLUSÕES A RESPEITO DA LINGUAGEM \textit{C}}
		Com relação à \textit{threads}, o limite máximo de \textit{threads} possíveis de serem executadas sem erro, na máquina anteriormente detalhada, é de 30.000 \textit{threads}. É possível executar com até 33.010 \textit{threads}, porém, se esse valor for utilizado, apesar de nenhum erro ocorrer, a aplicação não consegue finalizar, ficando ``congelada''.
		
		Agora com relação aos processos, o limite máximo de processos possíveis de serem executados sem erro é de 10.330 processos.
		
		Finalmente, de forma geral e esperada, os algoritmos de ordenação executados em C apresentaram uma velocidade maior na ordenação das matrizes em comparação com os algoritmos executados em Java, tendo em mente que as matrizes usadas por ambas as linguagens possuíam o mesmo tamanho, embora não necessariamente possuíssem os mesmos dados.
		 
	\subsection{\normalsize CONCLUSÕES A RESPEITO DA LINGUAGEM \textit{Java}}
		Em Java, o limite máximo de \textit{threads} se mostrou superior em comparação à linguagem C, onde foi possível, embora com enorme perda de desempenho, executar 1.000.000 \textit{threads}, tendo em mente que também se mostrou possível executar um número maior de \textit{threads}, o que me leva a concluir que não existem limites para o número máximo de \textit{threads} em Java, ou que, pelo menos, esse limite é extremamente alto.
	
	\subsection{\normalsize CONCLUSÕES A RESPEITO DO ALGORITMO \textit{BUBBLE SORT}}
		O algoritmo apresenta um desempenho inferior ao \textit{Merge Sort} e, conforme o tamanho da matriz aumenta, o desempenho do algoritmo diminuí (considerando o mesmo número de \textit{threads}). 
		
		Agora, com relação a linguagem, ao manter o mesmo tamanho de matriz e o mesmo número de \textit{threads} em ambas as aplicações, notou-se um claro desempenho superior da linguagem C em relação à linguagem Java. Como, por exemplo, em uma matriz 100x100 e com 10 \textit{threads} executando, em C a ordenação da matriz levou 0.008282 segundos, enquanto em Java a mesma ordenação levou 0.02561 segundos. 
		
		Para maiores detalhes e mais casos de exemplo, favor ver os \textit{logs} de cada aplicação.
		
	\subsection{\normalsize CONCLUSÕES A RESPEITO DO ALGORITMO \textit{MERGE SORT}}
		O algoritmo apresenta um desempenho muito bom, mesmo com matrizes enormes e poucas \textit{threads}. Novamente a superioridade de desempenho da linguagem C sobre a linguagem Java se faz presente. Tomando como exemplo o caso relatado acima, enquanto a ordenação em C levou 0.001642 segundos, a ordenação em Java levou 0.008325 segundos, o que, apesar de parecer ser pouco, é uma diferença colossal de desempenho.
		
	\subsection{\normalsize DEMAIS CONSIDERAÇÕES}
		Como era de se esperar, quanto maior o tamanho da matriz, mais tempo se leva para ordena-lá. Além disso, percebeu-se que não vale a pena criar mais \textit{threads} do que o limite físico de núcleos da máquina, uma vez que o ganho de desempenho ao fazer isso, quando existe, não é muito significante, além, é claro, de poder ter o efeito oposto e acabar diminuindo o desempenho da aplicação.
		
		Finalmente, pode-se perceber que o desempenho da aplicação aumenta se a matriz tiver poucos linhas, independente do tamanho das colunas, em detrimento a uma matriz com muitas linhas mas colunas pequenas.
		
		Por fim, mas não menos importante, vale ressaltar que, caso reste alguma dúvida com relação ao desempenho das aplicações, basta ler os \textit{logs} anexados e, caso isso não seja suficiente, basta executar as aplicações anexadas você mesmo e fazer seus próprios testes e tirar suas próprias conclusões.
\end{document}
